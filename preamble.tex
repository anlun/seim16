\usepackage{fontspec}
\setmainfont[
 Path = ./ ,
 Extension = .otf ,
 Mapping=tex-text,
 Ligatures={TeX, Common},
 BoldFont = texgyretermes-bold,
 ItalicFont = texgyretermes-italic,
 BoldItalicFont = texgyretermes-bolditalic,
 SmallCapsFont = texgyretermes,
 SmallCapsFeatures = {Letters = SmallCaps}
]{texgyretermes}

\usepackage{polyglossia}
\setdefaultlanguage{russian}
\setotherlanguages{english}

\usepackage[backend=biber,style=ieee,refsection=section]{biblatex}    
% *** CITATION PACKAGES ***
%
% \usepackage{cite}
% cite.sty was written by Donald Arseneau
% V1.6 and later of IEEEtran pre-defines the format of the cite.sty package
% \cite{} output to follow that of the IEEE. Loading the cite package will
% result in citation numbers being automatically sorted and properly
% "compressed/ranged". e.g., [1], [9], [2], [7], [5], [6] without using
% cite.sty will become [1], [2], [5]--[7], [9] using cite.sty. cite.sty's
% \cite will automatically add leading space, if needed. Use cite.sty's
% noadjust option (cite.sty V3.8 and later) if you want to turn this off
% such as if a citation ever needs to be enclosed in parenthesis.
% cite.sty is already installed on most LaTeX systems. Be sure and use
% version 5.0 (2009-03-20) and later if using hyperref.sty.
% The latest version can be obtained at:
% http://www.ctan.org/pkg/cite
% The documentation is contained in the cite.sty file itself.

\usepackage{graphicx}
\usepackage{amsmath}
\usepackage{url}
\usepackage{array}
\usepackage[caption=false,font=footnotesize]{subfig}
\usepackage{lipsum}
% \usepackage[caption=false,font=footnotesize]{subfig}
% \usepackage{dblfloatfix}
% The latest version can be found at:
% http://www.ctan.org/pkg/dblfloatfix

\renewcommand{\thesubsection}{\Alph{subsection}.}
\renewcommand{\thesubsubsection}{\arabic{subsubsection})}
\renewcommand{\theparagraph}{\alph{paragraph})}


% The following code makes "fake small caps" which work fine
% for cyrillic symbols
% Borrowed from http://tex.stackexchange.com/questions/64582/faking-small-caps-in-xelatex?rq=1


\usepackage{expl3,xparse}

% turn expl3 space on: `:' and `_' are letters now and spaces
% are ignored. To insert a space use `~'.
\ExplSyntaxOn
% the internal command:
\cs_new:Npn \fakecaps:n #1
  {
    \addfontfeature{LetterSpace=10.0} {\tl_head:n { #1 }\kern 1pt}{\addfontfeature{Scale=0.9}\uppercase{\tl_tail:n { #1 }}}
  }

% the document command:
\NewDocumentCommand\FakeCaps{m}
  {\fakecaps:n { #1 } }

% turn expl3 space off again:
\ExplSyntaxOff

\makeatletter
\newcommand\subparagraph{%
  \@startsection{subparagraph}{5}
  {\parindent}
  {3.25ex \@plus 1ex \@minus .2ex}
  {-1em}
  {\normalfont\normalsize\bfseries}}
\makeatother
\usepackage[explicit]{titlesec}
\let\subparagraph\relax
\titleformat{\section}[hang]{\normalfont\normalsize\centering}{\thesection.}{0.5em}{\FakeCaps{#1}}

\makeatletter
\def\russian@capsformat{}
\makeatother
